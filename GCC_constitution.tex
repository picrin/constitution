\documentclass{report}

% Two Societie's names
\newcommand{\society}{Glasgow Computing Club}
\newcommand{\shortsociety}{GCC}
\newcommand{\bearersNo}{five}

% Set font
\renewcommand{\familydefault}{\sfdefault}


% Society name
\begin{document}

% Title page
\title{The \society{} Constitution}
\author{Authored by the \society{} Committee}
\date{\today}
\maketitle{}

\tableofcontents
\newpage

% --------------------------------------------------
\chapter{Preamble}
% --------------------------------------------------

	\section{Name}

		The name of the society shall be the \textbf{\society{}} (hereafter referred to as \textit{the Society}) and shall be abbreviated as \textbf{\shortsociety{}}.

	\section{Equal Opportunities}

		The Society will provide and promote equal opportunities, whatever a person’s race, operating system they use, colour, ethnicity or national origin, level in World of Warcraft, religion, beliefs, sex, age, sexuality, gender identity, HIV status, physical or mental disability, state of health, appearance, marital status or family circumstances. 

	\section{Aims}

		The society's aims and purposes shall be:

		\begin{enumerate}
			\item{to create a platform for software and hardware development, IT startups and tech enterprise,}
			\item{to broaden interest in Computing Science and related disciplines,}
			\item{to informally introduce younger students to team programming and to give an opportunity to older students to learn team programming management,}
			\item{to propagate open source movement,} 
			\item{and to collaborate on tech-related projects in:

				\begin{enumerate}
					\item{Web Development,}
					\item{Mobile Development,}
					\item{Open Hardware}
					\item{Scientific Modelling,}
					\item{Security and Cryptography,}
					\item{and Game Development}
				\end{enumerate}	
			}

		\end{enumerate}

	\section{Activities}

		The society will realise its aims and purposes through the following activities opened to all its members and optionally to wider public:

		\begin{enumerate}
			\item{collaboration on programming projects by means of Distributed Version Control Managers,}
			\item{annual hackathon in semester 1,}
			\item{talks, workshops and other educational events engaging students, members of Computing Science Department staff and Scottish IT industry,}
			\item{weekly non-academic social events, such as Lan Parties or enjoying of beverages.}
			\item{gathering into special interest groups to collaborate on tech-related projects.}
			\item{an annual introductory event for 1st year students.}
		\end{enumerate}

% --------------------------------------------------
\chapter{Members and Committee}
% --------------------------------------------------

\section{Membership}

	\begin{enumerate}
		\item{Full membership shall be open to registered students of University of Glasgow only.}
		\item{Non-students may join as Associate members.}i
		\item{Every member has to sign the Declaration of Responsibilty as provided by The School of Computing Science in order to be allowed access to the School's facilities.}
		\item{Associate members (non-students) shall not account for more than 20\% of the total membership.}
	\end{enumerate}

\section{Governance}

	\begin{enumerate}
		\item{The Society shall be under the control of the membership.}
		\item{The Society shall have \bearersNo{} office-bearers: President, Vice President, Treasurer, Secretary, Communication Officer and Social Events Organiser.}
		\item{The Committee shall consist of the President, Vice President, Treasurer and Secretary and the representatives of all special interest groups.}
		\item{All \bearersNo{} office bearers shall be registered students at the University of Glasgow and shall not have opted out of SRC representation under the Education Act of 1994.}
		\item{Any full member of the society is entitled to stand for the office bearer positions.}
	\end{enumerate}
	
\section{Committee and Responsibilities}

	\begin{enumerate}
		\item{The President:
			\begin{enumerate}
				\item{shall be the spokesperson for the Club/Society.}
				\item{shall chair meetings of the Committee unless unavailable }
				\item{shall co-ordinate the work and activities of the Committee.}
			\end{enumerate}
		}
		\item{The Vice-President:
			\begin{enumerate}
				\item{shall chair meetings of the Committe and perform any other President’s duties in the case of President's absence}
				\item{should be the president's advisor in any important decisions}
			\end{enumerate}
		}
		\item{The Treasurer:
			\begin{enumerate}
				\item{Shall maintain a record of the income and expenditure of the Club/Society.}
				\item{Shall be responsible for preparing the accounts and shall keep bank statements.}
			\end{enumerate}
		}
		\item{The Secretary:
			\begin{enumerate}
				\item{shall keep minutes of AGMs, EGMs and Committee Meetings}
				\item{shall maintain an up to date membership list.}
			\end{enumerate}
		}
		\item{Entire Committee, by decision of majority:
			\begin{enumerate}
				\item{shall approve the creation of special interest groups based on critical mass, indication of support and checks on any ethical issues.}
				\item{Board will decide annualy which groups will persist into new year based on attendance at meetings/evidence of progress.}
				\item{shall maintain an up to date membership list.}
			\end{enumerate}
		}
	\end{enumerate}

% --------------------------------------------------
\chapter{General Meetings and Elections}
% --------------------------------------------------


\section{General Meetings}
	\subsection{Annual General Meeting}
		\begin{enumerate}
			\item{The purpose of an Annual General Meeting (AGM) is to enable all members of the Club/Society to discuss the previous year’s strengths and weakness, to discuss the future direction of the Club/Society, to make any constitutional amendments, to elect the office bearers and to present the Club/Society’s financial statement. The membership must be given a financial statement on the year’s accounts. }
			\item{The General Meeting shall take place annualy in January or February every year.}
			\item{Office bearers shall be elected by the membership at the AGM or at an EGM called to elect a replacement during the event of a vacancy. All office-bearers must initially be elected by the board and membership at the AGM. Those \bearersNo{} office bearers named above must be elected by the membership, even in the event of a vacancy. It is advised that any prospective office bearers would have shadowed previous office bearers.}
			\item{The Secretary shall give fourteen days notice of an AGM.}
		\end{enumerate}
	\subsection{Extraordinary General Meeting}
		\begin{enumerate}
			\item{An Extraordinary General Meeting (EGM) shall be called either by the Committee or by submission of a written request by 10\% of the membership. }
			\item{The Secretary shall give five working days notice of an EGM.}
			\item{The quoracy of an EGM or AGM is 10\% of the full membership.}
			\item{Any member who is a registered student shall have full voting rights at the AGM or an Extraordinary General Meeting (EGM) of the Club/Society.}
			\item{At any General Meeting of the Society, the weight assigned to the total vote of Associate members shall not exceed 10\% of the total voting members present.}
		\end{enumerate}
\section{Elections}
	Elections shall be by secret ballot.
% --------------------------------------------------
\chapter{Constitutional Amendments}
% -------------------------------------------------
		\begin{enumerate}
			\item{The membership may make constitutional amendments at the Annual General Meeting by decision of the membership majority provided that quorum is met.}
			\item{On receipt of a petition signed by 10\% of the membership, the Secretary shall give five working days notice of an Extraordinary General Meeting to consider submitted amendments.}
			\item{The Secretary shall give fourteen days notice of an AGM.}
		\end{enumerate}
\end{document}
